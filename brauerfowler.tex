\chapter{El teorema de Brauer--Fowler}

En este capítulo vamos a demostrar el teorema de Brauer--Fowler. El resultado es fundamental
en la clasificación de grupos simples finitos. Daremos dos demostraciones, una basada en 
teoría de caracteres y una demostración alternativa completamente elemental. 

\begin{theorem}[Brauer--Fowler]
\index{Teorema!de Brauer--Fowler}
Sea $G$ un grupo finito y simple y sea $x$ una involución. Si $|C_G(x)|=n$, entonces $|G|\leq (n^2)!$.	
\end{theorem}

La cota del teorema de Brauer--Fowler no es importante, ya que
para considerar una forma posible de atacar la clasificación de grupos simples 
solamente es necesario saber que existen  
finitos grupos simples finitos con un cierto centralizador de invouciones.

\begin{corollary}
...
\end{corollary}

Veamos un ejemplo sencillo que da una idea de cómo es que pueden clasificarse grupos simples
una vez que se tiene fija la estructura del centralizador de una involución. 

Vamos a demostrar
que si $G$ es un grupo simple finito y $x$ es una involución tal que $C_G(x)\simeq\Z/2$, entonces
$G\simeq\Z/2$. 

Supongamos que $\Irr(G)=\{\chi_1,\chi_2,\dots,\chi_k\}$. No conocemos la tabla de caracteres del grupo $G$ 
pero sabemos que...

\subsubsection*{Una demostración elemental del teorema de Brauer--Fowler}

Supongamos primero que existe un subgrupo propio $H$ de $G$ tal que
$(G:H)\leq n^2$. En ese caso, hacemos actuar a $G$ en $G/H$ por multiplicación a izquierda 
y tenemos un morfismo de grupos $\rho\colon G\to\Sym_{n^2}$. Como $G$ es un grupo simple, 
$\ker\rho=\{1\}$ o bien $\ker\rho=G$. Si $\ker\rho=G$, entonces $\rho(g)(yH)=yH$ para todo
$g\in G$ e $y\in G$, lo que implica que $g\in H$, una contradicción. Luego $\rho$ es inyectiva
y entonces $G$ es isomorfo a un subgrupo de $\Sym_{n^2}$. En particular, $|G|$ divide a $(n^2)!$

Sea $X$ la clase de conjugación de $x$. Para $g\in G$ definimos
\[
J(g)=\{z\in X:zgz^{-1}=g^{-1}\}.
\]
Primero veamos que $|J(g)|\leq|C_G(g)|$. La función $J(g)\to C_G(g)$, $z\mapsto gz$, está bien definida, 
pues 
\[
(gz)g(gz)^{-1}=g(xgx^{-1})g^{-1}=g^{-1}\in C_G(g),
\]
y es inyectiva, pues $gz=gz_1$ implica $z=z_1$.

Sea $\{(g,z)\in G\times X:zgz^{-1}=g^{-1}\}$.  
Como la función $X\times X\to J$, $(y,z)\mapsto (yz,z)$, 
está bien definida, pues $z(yz)z^{-1}=zy=(yz)^{-1}$, y es trivialmente una función inyectiva, 
tenemos entonces que
\[
|X|^2\leq |J|=\sum_{(g,z)\in J}1\leq\sum_{g\in G}|J(g)|=\sum_{g\in G}|C_G(g)|=k|G|,
\]
donde $k$ es la cantidad de clases de conjugación de $G$, 
pues $(g,z)\in J$ si y sólo si $z\in J(g)$. Luego $|G|\leq kn^2$, pues
\[
\left(\frac{|G|}{|C_G(x)|}\right)^2=|X|^2=\frac{|G|}{n^2}\leq k|G|.
\]

\begin{claim}
Existe alguna clase de conjugación que tiene $\leq n^2$ elementos.
\end{claim}

De lo contrario, si $C_1,\dots,C_k$ son las clases de conjugación de $G$ y 
$|C_i|>n^2$ para todo $i\in\{1,\dots,k\}$, entonces 
\[
|G|=\sum_{i=1}^k|C_i|>\sum_{i=1}^kn^2=kn^2\geq |G|,
\]
una contradicción. 

\begin{claim}
Existe un subgrupo $H$ de $G$ tal que $(G:H)\leq n^2$.
\end{claim}

Sea $C$ alguna clase de conjugación de $G$ tal que $|C|\leq n^2$ y sea $g\in C$.  
Entonces $H=C_G(g)$ es un subgrupo de $G$ tal que $(G:H)\leq n^2$. 
